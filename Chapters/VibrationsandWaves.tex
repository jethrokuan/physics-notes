\section{Vibrations and Waves}
Periodic motion, from masses on springs to vibrations of atoms, is one of the most important kinds of physical behavior.
\subsection{Hooke's Law}
One of the simplest types of vibrational motion is that of an object attached to a spring. If the spring is stretched or compressed a small distance $x$ from its unstretched or equilibrium position and then released, it exerts a force on the object.

It obeys the equation
$$F_{s}=-kx$$
k is a positive constant called the \textbf{spring constant}. It is important to note that the force exerted by the spring is always directed opposite the displacement of the object. Because the spring force always acts toward the equilibrium position, it is sometimes called a restoring force. A restoring force always pushes or pulls the object toward the equilibrium position.

Suppose the object is initially pulled a distance A to the right and released from rest. The force exerted by the spring on the object pulls it back toward the equilibrium position. As the object moves toward $x=0$ the magnitude of the force de- creases (because x decreases) and reaches zero at $x=0$. However, the object gains speed as it moves toward the equilibrium position, reaching its maximum speed when $x=0$. The momentum gained by the object causes it to overshoot the equilibrium position and compress the spring. As the object moves to the left of the equilibrium position (negative x-values), the spring force acts on it to the right, steadily increasing in strength, and the speed of the object decreases. The object finally comes briefly to rest at $x=-A$ before accelerating back towards $x=0$ and ultimately returning to the original position at $x=A$. The process is then repeated, and the object continues to oscillate back and forth over the same path. This type of motion is called simple harmonic motion. Simple harmonic motion occurs when the net force along the direction of motion obeys Hookes law-- when the net force is proportional to the displacement from the equilibrium point and is always directed toward the equilibrium point.

The following three concepts are important in discussing any kind of periodic motion:
\begin{enumerate}
\item The \textbf{amplitude} A is the maximum distance of the object from its equilibrium position. In the absence of friction, an object in simple harmonic motion oscillates between the positions $x=-A$ and $x=A$.
\item The \textbf{period} T is the time it takes the object to move through one complete cycle of motion, from $x=A$ to $x=-A$ and back to $x= A$.
\item The \textbf{frequency} f is the number of complete cycles or vibrations per unit of time, and is the reciprocal of the period ( $f=\frac{1}{T}$).
\end{enumerate}