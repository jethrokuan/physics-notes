\section{Newton's Laws of Motion}
\subsection{Newton's 3 Laws of Motion}
How can a tugboat push a cruise ship that's much heavier than the tug? Why is a long distance needed to stop the ship once it is in motion? These are all questions dealing with \textbf{dynamics}, the relationship of motion to the forces associated with it.

All the principles of dynamics can be wrapped up in a neat package containing three statements called \textbf{Newton's laws of motion}. They were first clearly stated by Sir Isaac Newton(1642 -- 1727). More information can be found about him \href{http://www.newton.ac.uk/newtlife.html}{here}.

\subsection{Force}
The concept of \textbf{force} gives us a quantitative description of the interaction between two objects or between an object and its environment.

\paragraph*{Types of Forces} \label{tension}
\begin{enumerate}
\item When a force involves direct contact between two objects, we call it \textbf{contact force}.
\item When an object rests on a surface, there is always a component of force perpendicular to the surface, called the \textbf{normal force}, denoted by $\vec{n}$.
\item There may also be a component of force parallel to the surface, called \textbf{friction or frictional force}, denoted by $\vec{f}$.
\item When a rope or cord is attached to an object and pulled, the corresponding force acting on the object is referred to as the \textbf{tension}, denoted by $\vec{T}$.
\item A familiar force we'll work with often is the gravitational attraction that the earth exerts on an object. This force is the object's \textbf{weight} which can be calculated by taking $mass \times weight$
\end{enumerate}

\paragraph*{Measuring force}
Force is a \textbf{vector} quantity; to describe a force, we need to describe the direction in which it acts as well as its magnitude -- the quantity that tells us ``how much" or ``how strongly'' the force pushes or pulls. The SI unit of the magnitude of force is \textbf{the \emph{newton}} abbreviated $N$. The official definition is based on the standard kilogram, which will be further explained later at section \ref{newton}.

\paragraph*{Resultant of Forces}
Experiment shows that when two forces $\vec{F}_1$ and $\vec{F}_2$ act at the same time on the same point, the effect is the same as the effect of a single force equal to the vector sum of the two forces. This vector sum is often called the \textbf{resultant} of the forces or \emph{net force}. The discovery that forces combine according to vector addition is of the utmost importance, in resolving forces later on in this chapter.

Everyone should already know this, if they have read up on \textbf{Scalars and Vectors}, not covered in this manuscript.

\subsection{Newton's First Law}
The fundamental role of a force is to \textbf{change the state of motion of the object on which the force acts}. Newton's first law of motion is as follows:
\begin{defi}[Newton's First Law of Motion]
Every object continues either at rest or in constant motion in a straight line, unless it is forced to change that state by forces acting on it.
\end{defi}

When \emph{no} force acts on the object, then the vector sum of forces on it is zero, and the object will continue in that state of motion. \textbf{An object acted on by no net force moves with constant velocity(which could be 0) and thus with zero acceleration.}

\paragraph{Inertia}
The tendency of an object to remain at rest, or to keep moving once it is set in motion, results from a property called \emph{inertia}. 

The quantitative measure of inertia is the \textbf{physical quantity called mass}, which we will discuss further in section \ref{mass}

\paragraph{Inertial Frames of Reference} \label{ifor}
In our previous discussions when we mentioned relative velocity, we mentioned ``relative to'' an object. This is a frame of reference. This concept also plays a important role in Newton's laws of motion.

Newton's law is valid is some frames of reference and not in others. A valid one is called the \textbf{inertial frame of reference}. 

Although it may seem that there's only one inertial frame of reference in the whole universe, but on the contrary, anything that is moving with the same constant velocity as that frame is also valid, and are all inertial frame of references.

There is no inertial frame of reference that is preferred over all others for formulating Newton's Laws. If one frame s inertial, then another frame moving relative to it at constant velocity is also inertial. Both the state and rest and state of uniform motion can occur when the vector sum of forces acting on the object is zero. Because Newton's first law can be used to describe and define what we mean by an \textbf{inertial frame of reference}, it is sometimes called the \emph{law of inertia}.

\subsection{Mass and Newton's Second Law}
We have learnt that an object acted on by a non-zero net force accelerates. We now want to know the relation of the acceleration to the force, which is what Newton's second law of motion states.

When a force with magnitude F acts on an object with mass m, the magnitude a of the object's acceleration is:
$$a=\frac{F}{m}$$

The Si unit of mass is the \textbf{kilogram}\label{mass}. We can use the standard kilogram, and the above equation to define the fundamental SI unit for force, the Newton, N\label{newton}:

\begin{defi}[Definition of the Newton] 
One newton is the amount of force that gives an acceleration of 1 meter per second squared to an object with a mass of 1 kilogram. That is,
$$1 N = (1kg) (1ms^{-2})$$
\end{defi}

We can use this definition to calibrate the spring balances and other instruments to measure forces.

Now we will move on to Newton's second law of motion, which states that:
\begin{defi}[Newton's Second Law of Motion] \label{def:N2L}
The vector sum (resultant) of all the forces acting on an object equals the object's mass times its acceleration (the rate of change of its velocity):
$$\sum \vec{F} = m \vec{a}$$
\end{defi}

Newton's second law is a fundamental law of nature, the \textbf{basic relation between force and motion}. The above equation is one that is vector, and has direction. We can separate it into their various components as such:

\begin{form}[Newton's Second Law Component Form] \label{form:N2L}
$$\sum F_x = m a_x \qquad and \qquad \sum F_y = m a_y$$
\end{form}

The definition \ref{def:N2L} and equation \ref{form:N2L} are only valid when the mass \emph{m is constant}. It's easy to think of systems whose mass change, like a leaking tank truck.

Like the first law, Newton's second law only holds true if they are in inertial frames of reference (Refer to section \ref{ifor}).

\subsection{Mass and Weight}
We have previously mentioned that the weight of an object is a force -- the force of gravitational attraction of the earth. The terms \emph{mass} and \emph{weight} are often mixed up and misused, interchanged in everyday conversation. The two terms are absolutely different, and it is essential to know the differences.

\emph{Mass} characterises the \textit{inertial} properties of an object. The greater the mass, the greater the force is required to cause a given acceleration. This is reflected from Newton's Second Law (refer to definition \ref{def:N2L})

\textbf{Weight is the force} exerted on an object by the gravitational pull of the earth or some other astronomical body. Everyday experience shows us that large object also have large weight.

\begin{defi}[Relation of mass to weight]
The weight of an object of mass $m$ then magnitude $w$ is equal to the magnitude of acceleration due to gravity, $g$ times the mass:
$$w=mg$$
Because weight is a force, it has a direction, and we can write it in a vector relation:
$$\vec {w} = m \vec{g}$$	
\end{defi}

\subsection{Newton's Third Law}
Again, experiments show that whenever two objects interact, the two forces they exert on each other are equal in magnitude and opposite in direction. This is what Newton's third law states.

\begin{defi}[Newton's Third Law]
For two interacting objects $A$ and $B$, the formal statement of Newton's third law is:
$$\vec{F}_{\text{A on B}} = - \vec{F}_{\text{B on A}}$$
Newton's own statement, translated from the Latin of the \emph{Principia}, is:

To every reaction there is always opposed and equal reaction; or, the mutual actions of two objects upon each other are always equal, and directed to contrary parts.
\end{defi}

\subsection{Application of Newton's Laws}
Newton's three laws of motion have already been clearly stated above, but how do we apply them? I shall answer this question in the following section. This whole chapter is about solving problems by applying the three laws.

\subsection{Equilibrium of a Particle}
We learnt previously that an object is in \textbf{equilibrium} when it is at rest or moving with constant velocity in an inertial frame of reference. \textbf{When an object is at rest or is moving with constant velocity in an inertial frame of reference, the vector sum of all the forces acting on it must be zero}.

\begin{form}[Necessary condition for equilibrium of an object]
For an object to be in equilibrium, the net force acting on it must be zero.
$$\sum \vec{F} = 0$$
This condition is sufficient only if the object can be treated as a particle, which we assume in the next principle and throughout the remainder of the chapter.
\end{form}

Similar to the way we treated Newton's Second Law(refer to definition \ref{def:N2L}) we can separate the forces into their individual components:

\begin{form}[Necessary condition for equilibrium of an object: component form]
$$\sum F_x = 0 \qquad and \qquad \sum F_y=0$$
\end{form}

\subsection{Application of Newton's Second Law}
We're now ready to discuss problems in \textbf{dynamics}. Refer to the definition (\ref{def:N2L}) and the component form (formula \ref{form:N2L}) for a recap of Newton's Second Law.

The steps to solving the question are:
\begin{enumerate}
\item Draw a sketch of the physical situation, and identify the moving object or objects to which you will apply Newton's Second Law.
\item Draw a free-body diagram for each chosen object, showing all the forces, and depicting their individual magnitudes accurately.
\item Show coordinate axes explicitly to avoid wrong signs.
\item Solve for the individual x and y components of the forces and write out equations for each chosen object.
\item Check if your result makes sense.
\end{enumerate}

\subsection{Contact forces and friction}
Whenever two objects interact by direct contact, frictional force is produced in the opposite direction. 

First, when an object rests or slides on a surface, we can always represent the contact force exerted by the surface on the object in terms of components of force perpendicular and parallel to the surface. We call the perpendicular component the \textbf{``normal force"}, denoted by $\vec{n}$ (Normal is a synonym for perpendicular). The component parallel to the surface is the \emph{friction force}, denoted by $\vec{f}$. By definition, $\vec{n}$ and $\vec{f}$ are always perpendicular to each other.

The magnitude $f_k$ of a kinetic-friction force usually increases when the normal force magnitude $n$ increases. Thus, more force is needed to slide a box full of books across the floor than to slide the same box when it is empty. In some cases, the magnitude of the sliding friction force $f_k$ is found to be approximately proportional to the magnitude n of the normal force. In such cases, we call the ratio $\frac{f_k}{n}$ the \textbf{coefficient of kinetic friction}, denoted as $\mu_k$.

\begin{defi}[Relation between kinetic--friction force and normal force]
When the magnitude of the sliding friction force $f_k$ is roughly proportional to the magnitude $n$ of the normal force, the two are related by a constant $\mu_k$ called the coefficient of static friction:
$$f_k = \mu_k n$$
Because $\mu_k$ is the ratio of two force magnitudes, it has no units.
\end{defi}

The numerical value of the coefficient of kinetic--friction for any two surfaces depends on the materials and the surfaces.

Friction forces may also act when there is \emph{no} relative motion between the surfaces of contact. If you try to slide a box of books across the floor, the box may not move at all if you don't push hard enough, because the floor exerts an equal and opposite friction force on the box. This force is called a \textbf{static-friction force}.

For a given pair of surfaces, the maximum value of $f_s$ depends on the normal force. In some cases, similar to that of kinetic friction, the maximum value of $f_s$ is approximately \emph{proportional} to n; we call this proportionality factor $\mu_s$, the coefficient of static friction.

\begin{defi}[Relation between normal force and maximum static--friction force]
When the maximum magnitude of the static-friction force can be represented as proportional to the magnitude of the normal force, the two are related by a constant $\mu_s$, called the coefficient of static friction:
$$ f_s \leq \mu_sn$$
\end{defi}

\subsection{Elastic Forces}
Again, experiments show that the amount of compression and stretching of a string is directly proportional to the magnitude of the force exerted on the spring. This proportionality was discovered by Robert Hooke (more information can be found about him \href{http://www.roberthooke.org.uk/intro.htm}{here}.) This simple equation is known as \emph{Hooke's law}.

\begin{form}[Hooke's Law]
For springs, the spring force $F_{spr}$ is approximately proportional to the distance $x$ by which the spring is stretched or compressed:
$$F_{spr} = -kx$$
\end{form}
In the above equation, $k$ is the positive proportionality constant called the force constant, or sometimes called the \textbf{spring constant}, of the spring.

This concludes the chapter, Newton's Law of Motion.