\section{Solids and Liquids}
There are four known states of matter: solids, liquids, gases, and plasmas. In the universe at large, plasmas's systems of charged particles interacting electromagnetically's are the most common. In our environment on Earth, solids, liquids, and gases predominate.

\subsection{States of Matter}
Matter is normally classified as being in one of three states: \textbf{solid, liquid, or gas}. Often this classification system is extended to include a fourth state of matter, called a \textbf{plasma}.

Solids can be classified as either crystalline or amorphous. In a crystalline solid the atoms have an ordered structure. For example, in the sodium chloride crystal (common table salt), sodium and chlorine atoms occupy alternate corners of a cube. In an \textbf{amorphous solid}, such as glass, the atoms are arranged almost randomly.

For any given substance, the liquid state exists at a higher temperature than the solid state. The intermolecular forces in a liquid aren't strong enough to keep the molecules in fixed positions, and they wander through the liquid in random fashion. Solids and liquids both have the property that when an attempt is made to compress them, strong repulsive atomic forces act internally to resist the
compression.

In the gaseous state, molecules are in constant random motion and exert only weak forces on each other. The average distance between the molecules of a gas is quite large compared with the size of the molecules. Occasionally the molecules collide with each other, but most of the time they move as nearly free, noninteracting particles. As a result, unlike solids and liquids, gases can be easily compressed. We'll say more about gases in subsequent chapters.

When a gas is heated to high temperature, many of the electrons surrounding each atom are freed from the nucleus. The resulting system is a collection of free, electrically charged particles -- negatively charged electrons and positively charged ions. Such a highly ionized state of matter containing equal amounts of positive and negative charges is called a \textbf{plasma}. Unlike a neutral gas, the long-range electric and magnetic forces allow the constituents of a plasma to interact with each other. Plasmas are found inside stars and in accretion disks around black holes, for  example, and are far more common than the solid, liquid, and gaseous states because there are far more stars around than any other form of celestial matter, except possibly \textbf{dark matter}.

\subsection{Deforming Solids}\label{deformingsolids}
While a solid may be thought of as having a definite shape and volume, it's possible to change its shape and volume by applying external forces. A sufficiently large force will permanently deform or break an object, but otherwise, when the external forces are removed, the object tends to return to its original shape and size. This is called \textbf{elastic behavior}.

For sufficiently small stresses, stress is proportional to strain, with the constant of proportionality depending on the material being deformed and on the nature of the deformation. We call this proportionality constant the elastic modulus:
\begin{defi}[Elastic Modulus]
$$\text{stress} = \text{elastic modulus} \times \text{strain}$$
\end{defi}

The elastic modulus is analogous to a spring constant. It can be taken as the stiffness of a material: A material having a large elastic modulus is very stiff and difficult to deform. There are three relationships having the form of Definition \ref{deformingsolids}, corresponding to tensile, shear, and bulk deformation, and all of them satisfy an equation similar to Hooke's law for springs:

\begin{form}[Hooke's Law]
$$F=kd x$$
\end{form}

\subsection{Young's Modulus: Elasticity in Length}

The word tensile has the same root as the word tension. The SI unit of stress is the newton per square meter ($N/m^2$), called the pascal (Pa):

\begin{defi}[Pascal]
$$1 Pa = 1N/m^2$$
\end{defi}

The tensile strain in this case is defined as the ratio of the change in length $d L$ to the original length $L_\circ$ and is therefore a dimensionless quantity. Using Definition \ref{deformingsolids}, we can write an equation relating tensile stress to tensile strain:

\begin{form}[Calculating Tensile Stress]
$$\frac{F}{A} = Y\frac{d L}{L_\circ}$$
\end{form}

In this equation, Y is the constant of proportionality, called \textbf{Young's modulus}. It could be solved for F and put in the form $\mathbf{F=kd L}$, where $k=YA/L_\circ$, making it look just like Hooke's law.

A material having a large Young's modulus is difficult to stretch or compress. This quantity is typically used to characterize a rod or wire stressed under either tension or compression. Because strain is a dimensionless quantity, Y is in pascals.

\subsection{Shear Modulus: Elasticity in Shape}
Another type of deformation is when an object is subjected to a force $\vec{F}$ \emph{parallel} to one of its faces while the opposite is held fixed by a second force, usually friction.

This kind of friction is usually called \textbf{shear stress}

We define this shear stress as:
\begin{defi}[Shear Stress]
$\frac{F}{A}$, the ratio of the magnitude of the parallel force to the area A being sheared. The shear strain is the ratio $\frac{d x}{h}$, where $d x$ is the horizontal distance the sheared face moves and $h$ is the height of the object.
\end{defi}

\begin{form}[Calculating Shear Stress]
$$\frac{F}{A}=S\frac{d x}{h}$$
\end{form}

Under this kind of stress, there is no volume deformation.

\subsection{Bulk Modulus: Volume Elasticity}
Suppose that the external forces acting on an object are all perpendicular to the surface on which the force acts and are distributed uniformly over the surface of the object. This occurs when an object is immersed in a fluid. An object subject to this type of deformation undergoes a change in volume but no change in shape. 

\begin{defi}[Bulk Modulus]
The volume stress $d P$ is defined as the ratio of the magnitude of the change in the applied force $d F$ to the surface A
\end{defi}
\begin{form}[Calculating Bulk Modulus]
$$d P=-B\frac{d V}{V}$$
\end{form}

\subsection{Density and Pressure}
Equal masses of aluminium and gold have different volume distance.

This is due to the difference, the concept of density
\begin{defi}[Density]
The density $\rho$ is equal to the mass divided by its volume
$$\rho=\frac{m}{V}$$
\end{defi}

Another important aspect of this subject is the specific gravity, which is defined as follows:

\begin{defi}[Specific Gravity]
The specific gravity of a substance is the ratio of its density to the density of water at $4^\circ C$, which is $1.0 \times 10^3 kg m^{-3}$
\end{defi}

We also define pressure here, just in case (:
\begin{defi}[Pressure]
If $F$ is the magnitude of a force exerted perpendicular to a given surface of area A, then the pressure P is equal to the Force divided by the Area,
$$P=\frac{F_{perpendicular}}{A}$$
SI unit: Pascal (Pa)
\end{defi}

The formula for calculating the pressure under the water can be given by:
\begin{form}[Calculating Pressure at different depths]
$$P=\rho g h$$
Where P is the pressure, $\rho$ is the density of the fluid, $g$ is the gravitational acceleration, and $h$ is the depth under water.
\end{form}

Because pressure in a fluid depends on depth, and on the value of $P_o$, any increase in pressure at the surface must be transmitted to every point in the fluid.

French scientist Pascal took this understanding, and came up with a principle:
\begin{defi}[Pasca''s Principle]
A change in pressure applied to an enclosed fluid is transmitted undiminished to every point of the fluid
\end{defi}

This principle is largely applied in hydraulic presses. Because fluid pressure is always equal, then let there be two pressure at different points. $P_1 = P_2$ at all times. Then it follows that $F_1/A_1 = F_2/A_2$, then the magntiude of $\vec{F_2}$ is larger than $\vec{F_1}$ by a factor of $A_2/A_1$. This is idea used in lifting heavy items.

\subsection{Buoyant Forces and Archimedes' Principle}
A fundamental principle affecting objects submerged in fluids was discovered by the Greek Mathematician Archimedes. Archimedes' principles can be stated as follows:

\begin{defi}[Archimedes Principle]
Any object completely or partially submerged in a fluid is buoyed up by a force with magnitude \textbf{equal to the weight} of the fluid displaced by the object
\end{defi}

Water provides partial support to any object placed in it. This force is called the \textbf{buoyant force}.

We can actually derive this from the first equation, on deriving pressure from . Because horizontal forces cancel, and in the vertical direction $P_2A$ acts upwards on the bottom of the block of fluid and $P_1A$ and the gravity force on the fluid, Mg, act downwards, giving
$$B=P_2A-P_1A = Mg$$

The buoyancy force can be identified as a difference in pressure equal in magnitude to the weight of the displaced fluid.

Using the definition of density, we get the equation:
$$B=\rho_{fluid}V_{fluid}g$$

where $\rho_{fluid}$ is the density of the liquid
and $V_{fluid}$ is the \textbf{volume of fluid displaced}

It is also good to note the different case breakdowns in the analysis of buoyant force:
\begin{enumerate}
\item Fully submerged objects
\subitem When an object is completely submerged in the fluid, the the volume of water displaced is the volume of the object itself, and thus, the Buoyant force experienced, $B=\rho_{fluid}V_{obj}g$ where $V_{obj}$ is the volume of the object.
\subitem As a result, if the density of the object is less than the density of the fluid, the net force exerted on the object is positive(upward) and the object accelerates upward.
\subitem If the density of the object is greater than the density of the fluid, it will have a negative net force, and accelerate downwards
\item Floating objects
\subitem We assume that the object is in static equilibrium floating in a fluid.
\subitem The upward buoyant force is balanced by the downward force of gravity acting on the object. If $V_{fluid}$ is the volume of liquid displaced by the object, then the magnitude of the buoyant force is given by $B=\rho_{fluid}V_{fluid}g$. Because the weight of the object is $w=mg=\rho_{obj}V_{obj}g$, and because w = B, it follows that:
$$\frac{\rho_{obj}}{\rho_{fluid}}=\frac{V_{fluid}}{V_{obj}}$$
\subitem Note that buoyant fore of the air is neglected, but is insignificant, since the density of air is low at sea level.
\end{enumerate}

\subsection{Fluids in Motion}
When a fluid is in motion, its flow can be characterized in one of the two ways. The flow is said to be \textbf{streamline}, or \textbf{laminar}

In discussions of fluid flow, the term \textbf{viscosity} is used often, to describe the degree of internal friction of the fluid.

The most important part of this section is the \textbf{Law of Continuity} and I guess I shall go straight to the point.

\subsection{Law of Continuity}\label{lawofc}
This law is usually used and applied in tubes containing fluid.

For the case of an incompressible liquid, which is often assumed to be so, the equation that is always followed is:
\begin{defi}[Law of Continuity]
$$A_1v_1 = A_2v_2$$
Where A represents the cross-sectional area of the tube, and v representing the velocity of the fluid
\end{defi}
In other words, the condition $Av$ is always a constant throughout a closed tube.

Following this discovery, Scientist Bernoulli also discovered an amazing finding. He managed to relate the pressure of a flujid to its speed and elevation. He then called this equation the \textbf{Bernoulli's equation}.

\subsection{Bernoulli's Equation}
As stated earlier, this equation relates the pressure of a fluid to the speed and elevation.

I shall go through the derivation of this equation with you. Before this, I must make some assumptions. As before, we assume that the fluid is incompressible, non viscous and flows in a steady state manner.

Consider the flow of a liquid in a non-uniform tube, and consider the time period $d t$. The force on the lower end of the tube is $P_1A_1$ (recall that $P=\frac{F}{A}$ by definition), where $P_1$ is the pressure at the lower end.

Then the work done on the lower end of the fluid by the fluid behind it is:
$$W_1 = F_1d x_1 =P_1V$$
where V is the volume of the initial region of the fluid. In a similar manner, the work done on the upper region at the time $d t$ is
$$W_2 = -P_2A_2d x_2 = -P_2V$$
This is volume of the liquid is the same because by the previous equation, the law of continuity(Refer to \ref{lawofc})

The net work done by these forces in the time $d t$ is
$$W_{fluid} = P_1V - P_2V$$

Part of this work goes into changing the fluid's kinetic energy, and part goes into changing the gravitational potential energy of the fluid--earth system.

If $m$ is the mass of the fluid passing through the pipe in the time interval $d t$, then the change in kinetic energy of the fluid is
$$d KE = \frac{1}{2}mv_2^2 - \frac{1}{2}mv_1^2$$

The change in the gravitational potential energy is
$$d PE = mgy_2 - mgy_1$$

Because the net work done by the fluid on the segment of fluid shown changes the KE and GPE only, then:

$$W_{fluid} = d KE +d GPE$$
$$P_1V-P_2V = \frac{1}{2}mv_2^2-\frac{1}{2}mv_1^2+mgy_2 - mgy_1$$

If we divide each term by V and recall that $\rho=\frac{m}{V}$, then we get
$$P_1-P_2 = \frac{1}{2}\rho v_1^2 -\frac{1}{2}v_1^2$$

Rearranging, we get:
$$P_1 + frac{1}{2}\rho v_1^2 + \rho g y_1= P_2 + \frac{1}{2}\rho b_2^2 +\rho g y_2$$

This is \textbf{Bernoulli's equation}, often expressed as
\begin{form}[Bernoulli's equation]
$$P+\frac{1}{2}\rho v^2+\rho gy = constant$$
\end{form}

The proper definition of the equation is:
\begin{defi}[Bernoulli's equation]
Bernoulli's equation states that the sum of the pressure P, the kinetic energy per unit volume, $\frac{1}{2}\rho v^2$, and the potential energy per unit volume, $\rho gy$ has the same value at all points along a streamline
\end{defi}

\subsection{Surface Tension}
This section will be ignored for the moment, since it is not in syllabus.

This concludes our chapter on Solids and Fluids.