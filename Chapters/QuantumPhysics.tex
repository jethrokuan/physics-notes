\section{Quantum Physics}
This branch of Physics describe the behavior and energy and matter at the atomic and subatomic scale. \footnote{\href{http://en.wikipedia.org/wiki/Introduction_to_quantum_mechanics}{Quantum mechanics - Wikipedia, the free encyclopedia}}

\subsection{Photoelectric Experiment}
\label{photoelectricexp}
It was Albert Einstein who conducted this experiment. If you are interested in his works, you can find more information about him \href{http://en.wikipedia.org/wiki/Albert_Einstein}{here}. I shall continue explaining the Photoelectric experiment.

The experiment's purpose was to \textbf{study the emission of electrons from a metal surface which is irradiated with light and experimentally determine the value of Planck's constant (Refer to section \ref{planckconst}) ``$h$'' by making use of the spectral dependency of the photoelectric effect}.

Max Planck was a brilliant physicist, who was considered to be the \textbf{founder of the quantum theory}. Planck was awarded the Nobel Prize in Physics in 1918. In 1894 Planck turned his attention to the problem of black-body radiation. He had been commissioned by electric companies to create maximum light from lightbulbs with minimum energy. Many argue that Einstein, who discovered the photoelectric effect should be given credit for quantum physics because Planck did not understand in a deep sense that he was "introducing the quantum" as a real physical entity. But it was true that Planck was the one who first proposed the idea and deserved credit too.

\subsection{Classical Wave Theory}
The classical wave theory states that:
\begin{itemize}
\item Photoelectrons emitted for radiation are in all wavelengths
\item Maximum $K.E.$ of photoelectron depends on light intensity, and is independent of frequency
\item Measurable time lag between emission of electrons(photons) $\Longrightarrow$ Need to gain enough energy
\end{itemize}

The photoelectric effect \textbf{does not} agree with this theory. A new theory had to be created to explain the effect.

\subsection{Quantum theory}
The idea suggests that:
	\begin{enumerate}
	\item electromagnetic energy is particulate in nature $\Longrightarrow$ transmitted through quanta\footnote{A fundamental particle, building block of protons and neutrons, as well as all other hadrons and mesons}
	\item The packets of energy are called photons, which travel in \textbf{only one direction}.
	\item The amount of energy E contained in each quantum is directly proportional to the frequency:
	\begin{form}[Einstein's Photoelecric equation -- Basic]
	$$E=hf$$
	$E=$ energy contained, 
	$h =$ Planck's constant$ = 6.63 \times 10^{-34} J s$\label{planckconst}, and
	$f =$ frequency of radiation
	\end{form}
	\end{enumerate}
\subsection{Einstein's Photoelectric equation}
This section requires thorough knowledge in the photoelectric experiment (Refer to \ref{photoelectricexp}
For a photoelectron to be emitted, sufficient energy is needed. Every material or metal has got a \textbf{characteristic work function, $\phi$}. This is the \textbf{minimum amount of energy needed to liberate an electron from its surface.} 

If the photoelectron is emitted without any further interactions with other atoms, its kinetic energy would be at maximum. Thus, Einstein proposed this equation:
\begin{form}[Einstein's Photoelectric Equation -- work function]
$$hf = \phi + KE_{max}$$
Since from section , $KE_{max} = \frac{1}{2}mv^2_{max}$, we get:
$$hf=\phi + \frac{1}{2}mv^2_{max}$$
\end{form}
The two equations above are known as \textbf{Einstein's Photoelectric equation}.

The maximum $KE$ is measured by measuring the \textbf{stopping potential }$V_s$, of the collector that is made negative with respect to the emitter. This represents the minimum negative potential required to stop even the most energetic electron from reaching the collector(there will be no current in the ammeter).

Thus, the stopping potential and maximum kinetic energy are related by the following equation:
$$\frac{1}{2}mv^2=eV_s$$

Thus, the photoelectric equation can also be rewritten as $hf=\phi + eV_s$.
\subsection{The Wave-Particle Duality}
Light can behave as a particle(from the photoelectric effect, refer to \ref{photoelectricexp}) when it interacts with matter or as a wave(reflection, refraction and interference(Young's Double Split Experiment)). This shows the wave particle duality of electromagnetic energy.

De Broglie suggested that matter might also exhibit this duality and have wave properties.

He suggested that for a particle of momentum $p=mv$, which exhibits wave behavior, it will have an associated wavelength, $\lambda$, given by 
$$\lambda = \frac{h}{p} = \frac{h}{mv}$$
Where:\\
$h =$ Planck's constant = $6.63\times10^{-34}Js$, 
$p = $ momentum, 
$m =$ mass, 
$v =$ velocity

This is also known as the \textbf{de Broglie principle}.

\subsection{Electron Diffraction}
When a beam of electron falls on a thin layer of graphite atoms, the electrons disperse and their pattern is captured on a coated fluorescent screen. The pattern is identical to that obtained due to the interference of electromagnetic radiation. This is known as electron diffraction phenomenon.

\subsection{Bohr's Atomic Model}
Several models of atom suggest that it consist of a nucleus with electrons orbiting around the nucleus. Maxwell's theory of electromagnetic radiation meant that electrons would emit radiation and lose energy while orbiting.

Niel Bohr had a different model. His model stated that:
\begin{itemize}
\item electrons within atoms could exist in stable energy states or levels without emitting radiation
\item electrons occupy the lowest energy level(known as the ground state) so that the energy of the atom is at the lowest and is the most usable
\item electrons would absorb energy in quanta of certain definite amounts when bombarded by atoms or other electrons
\item electron will jump to a higher level on gaining its quantum of energy, thus being in an\textbf{ excited state(excitation)}. If quantum of energy is sufficiently high, the electron might escape from \textbf{atom surface(ionisation)}.
\end{itemize}
\subsection{Line Spectra}
When a gas is energised sufficiently by being heated or having an electric current passed through it, it emits visible light. Using a spectrometer, an emission spectrum is observed.

Gases like neon and hydrogen give a spectrum that consists of a few colours only. A line spectrum is then observed. The term ``line spectra" indicates that only certain frequencies are present. These lines are evidence of ``quantised" energy levels in an atom.\textbf{ A line spectra is the characteristic of each element.}

When a gas is energised, the electron jumps to a higher energy level according to the amount of energy received. The atom would not be stable. The electron would then fall back to its original energy level, and in the process emit radiation for the amount of energy.

\subsection{Energy levels and Line Spectrum of Hydrogen}
Hydrogen gas is filled inside a tube and bombarded with electrons. The resulting spectrum is observed using a spectrometer.

The energy level diagram and the corresponding line spectrum is shown below:

\begin{enumerate}
\item All the energy levels have negative energy values. Energy of an electron at rest outside atom is taken to be 0.
\item The energy value of a given level may be expressed in units of an $eV$ (electronvolt). $1 eV = 1.6 \times 10^{-19}J$.
\item The frequency of the photon or radiation emitted when a electron falls from energy level $E_2$ to $E_1$is given by:
$$hf=E_2- E_1$$ 
\end{enumerate}

\subsection{Features of line spectra}
Consider the Balmer series $(E_n \rightarrow E_2)$ for the hydrogen atom:
\begin{enumerate}
\item There is an infinite number of spectrum lines in the Balmer series
\item Spacing between adjacent lines increase with increasing wavelength( or decrease with increasing frequency)
\item The series limit of $365.6 nm$ is the shortest possible wavelength emitted, corresponding to the transition $(E_{\infty} \rightarrow E_2)$
\item Generally, spectral lines corresponding to longer wavelengths are more intense because these transitions are more common.
\end{enumerate}
\subsection{Emission Line Spectra}
When a gas is heated or bombarded by electrons, the electrons in the gas atoms are excited to higher energy level. They will only remain there momentarily before emitting a photon and moving to lower energy state. This causes a series of lines in a spectrum, which is called the emission line spectrum. It consists of a series of separate bright line of definite wavelength on a dark background.

\subsection{Blackbody Radiation}
\subsection*{The Stefan--Boltzmann Equation}

Every object at a temperature greater than 0K emits electromagnetic radiation consisting of a mixture of wavelengths. The rate of energy at which energy is radiated is given by the Stefan-Boltzmann equation:
\begin{form}[The Stefan--Boltzmann Equation]
 $$P=e\sigma AT^4$$
 Where $P =$ total power radiated in all wavelengths, \\
 $T =$ absolute(Kelvin) temperature of the object, \\
 $\sigma =$ Stefan's constant $= 5.67 \times 10^{-8} W m^{-2} K^{-4}$\\
 $A =$ Surface area of radiating object\\
 $e =$ emissivity
\end{form} 
 The emissivity e:
 \begin{itemize}
 \item is characteristic of the surface of the radiating surface
 \item takes on a value \textbf{between 0 and 1}, i.e. $0\le \sigma \le1$
 \begin{itemize}
 \item Very black and rough surfaces have a value closer to 1.
  \item White and smooth surfaces have a value closer to 0.
 \end{itemize}
  \item Is somewhat dependent on the temperature of the body
 \end{itemize}
 
 \subsection{Uncertainty Principles}
 
Uncertainty Principles are well-popularised iconic principles of quantum theory due to its major departure from the determinism of classical physics. The program of quantum mechanics is very different -- the way physical observables are understood and calculated involves a \textit{probabilistic} interpretation of many physical phenomena around us. Thus, uncertainty principles -- such as those involving position--momentum, energy--time,  often become the first striking features of quantum physics.
 
 Imagine that we have to measure the position and momentum of a particle at the same time. Classically, the only limitation to the precision at which both its position and momentum can be measured is due to technology and the method of measurement. However, it turns out that in quantum mechanics, this fuzziness in spatial position and momentum is \textit{fundamental}. \href{http://nobelprize.org/nobel_prizes/physics/laureates/1932/heisenberg-bio.html}{Werner Heisenberg} derived this notion in 1927, now popularly known as the \textbf{\href{http://scienceworld.wolfram.com/physics/UncertaintyPrinciple.html}{Heisenberg's uncertainty principle}} which states:
 
 If a measurement of position is made with precision $d x$ and a simultaneous measurement of momentum in the $x$--direction, is made with precision $d p$ then the product of the two uncertainties can never be smaller than $\frac{h}{2}$ i.e.
 
 \begin{form}[Heisenberg's Uncertainty Principle --- position--momentum]
$$\Delta x \Delta p \ge \frac{\hbar}{2}$$
\end{form}

This uncertainty principle deals with position and momentum. Now I shall introduce another uncertainty principle, the energy-time uncertainty principle. From the name of the principle it is obvious that the latter deals with energy and time. The formula for the equation is:
\begin{form}[Heisenberg's Uncertainty Principle -- energy--time]
$$\Delta E \Delta t \ge \frac{\hbar}{2}$$
\end{form}

Now the usual context in which one uses the energy-time uncertainty principle is that of electromagnetic radiation involved in atomic processes. Suppose we wish to measure the energy E emitted/absorbed during the time interval $d t$, the finite time of the radiation process limits the accuracy with which we can determine the frequency -- and hence the energy $E$ of the waves. Thus, the energy-time uncertainty principle is a statement about the uncertainty in an energy measurement taken within the time interval $d t$.
