\section{Matter}
 In this topic, I will be going through three main points:
 \begin{enumerate}
 \item Properties of nuclei
 \item Nuclear Reactions
 \item Radioactivity
 \end{enumerate}
 
 \subsection{Properties of Nuclei}
 \subsection*{Rutherford's $\alpha$-particle Scattering Experiment}\label{rutherford}
 In 1911, Ernest Rutherford and his students Hans Geiger and Ernest Marsden used a beam of positively charged alpha particles to fire on a thin gold foil, trying to break open an atom to identify any particle that emerged.
 
 (Note: An $\alpha$-particle is a Helium nucleus stripped of its electrons, hence having a positive charge of +2e, where e is the fundamental charge)
 
 \textbf{Results:}
 \begin{itemize}
 \item Most of the particles went straight through or were scattered through very small angles
 \item BUT a very small fraction(less than 1\%) of the particles were scattered through very large angles, some of which were close to $180^0$.
 \end{itemize}
 
 \subsection{Atomic Structure}
 \begin{enumerate}
 \item Every atom has a central, positively charged nucleus that is very small but negative.
 \item The nucleus is made up of positive protons and neutral neutrons.
 \item Electrons revolve around the nucleus.
 \item There is a vast amount of empty space within an atom.
 \end{enumerate}
 
 \begin{tabular}{| l | l | l | l |}
 \hline
  & Proton & Neutron & Electron \\ \hline
 Mass(kg) & $1.673 \times 10^{-27}kg$ & $1.675 \times 10^{-27} kg$ & $9.109 \times 10^{-31}kg$\\ \hline
 Charge(C) & $+1.602 \times 10^{-19}C$ & 0 & $-1.602 \times 10^{-19}C$ \\ \hline
 Mass (u) & $1.007276 u$ & $1.008665u$ & $0.0005486u$ \\ \hline
 \end{tabular}
 
 u is the atomic mass unit, defined to be $\frac{1}{12}$ of the mass of a $^{12}C$ atom and is equivalent to $1.6605402 \times 10^{-27} kg $. At the atomic and subatomic level, u is used as the standard unit for masses instead of kg.
 
 Use Avogadro's number to show that $u= 1.66 \times 10^{-27}$.

\subsection{Properties of Nuclei}
\begin{itemize}
\item The nucleus contains protons and neutrons -- collectively referred to as nucleons
\item The nucleus is very small as compared to the size of the atom
\begin{itemize}
\item Nuclear diameters  are around $10^{-15} m$
\item Atomic diameters  are around $10^{10} m$
\end{itemize}
\item Over 99.9\% of the mass of an atom is in its nucleus
\item All nuclei(except for hydrogen nucleus) are composed of two types of particles, protons and neutrons. In representing nuclei, scientists use symbols in the general form of $^A_Z X$, where \textbf{X} is the chemical symbol for the element and:
\begin{description}
\item[A - mass number] The number of nucleons (neutrons $+$ protons) in the nucleus. Also known as nucleon number.
\item[Z - atomic number] The number of protons in the nucleus. It is also referred to as the proton number. It defines the chemical characteristics of an atom and the place of the element in the periodic table.
\item[N - neutron number] It can be obtained by taking $A - Z$.
\end{description}
\end{itemize}
\subsection{Isotopes}
\begin{defi}[Isotopes]
Isotopes are atoms that have the same number of protons but different number of neutrons
\end{defi}
Atoms that have the same number of protons will also have the same number of electrons. Since the number of electrons determines the chemical properties of an element, isotopes therefore exhibit the same chemical properties.

\textbf{Some examples of isotopes:}

\begin{tabular}{|l l | l  l| l  l|}
\hline
1. & $^1_1 H$ -- Hydrogen & 2. & $^{12}_6 C$ -- Carbon--12 & 3. & $^{35}_{12} Cl$ -- Chlorine--35\\[5pt]
 & $^2_1 H$ -- Deuterium & & $^{14}_6 C$ -- Carbon--14 & & $^{37}_{12} Cl$ -- Chlorine--37\\  [5pt]
 & $^3_1 H$ -- Tritium & & & &\\
 \hline
\end{tabular}

A \textbf{radioisotope} is an isotope of an element that is \textbf{radioactive}.

A nuclide is any particular atomic nucleus with a \textbf{specific} atomic number $Z$ and mass number $A$. E.g. $^{12}_6 C$ and $^{56}_{26}Fe$. Isotopes and nuclides having the same atomic number.

\subsection{Nuclear Stability (The Nuclear Force)}

Why does a nucleus remain intact and not burst apart?

There must be an even stronger force (which we call ``nuclear force") and attractive in nature at this distance ($10^{-15}m$). "Nuclear force" Is the strongest force in nature and has the following characteristics:

\begin{enumerate}
\item It is independent of electric charge
\item It is a very short range attractive force. This means that it is very strong when nucleon--nucleon distance is approximately $10^{-15}m$ , and decreases to zero at large distances($> 10^{-15}m$)
\end{enumerate}

For a nuclide to be stable, there must be sufficient neutrons to ``glue" the protons together.

Every proton repels every other proton, i.e. a proton is repelled by all other protons via electric force. But they are attracted by neighbouring nucleons via a strong nuclear force. Hence, for heavier nuclides, more neutrons are needed to separate the protons further apart.
As the atomic number increases, the line of stability deviates upwards from the line $N=Z$. This can be understood by recognising that, as the number of protons increases, the strength of the Coulomb force increases, which tend to break the nucleus apart. As a result, more neutrons are needed to keep the nucleus stable because neutron experience only the attractive nuclear force.

\subsection{Einstein's Mass--Energy Equivalence}

In 1905, Einstein showed from his theory of relativity that mass and energy are equivalent.

As a result of Einstein's theory, the separate conservation principles of mass and energy can be unified as the principle of mass-energy. Mass can be 	`created' or `destroyed' but when this happens, an equivalent amount of energy simultaneously vanishes or comes into being.
\begin{form}[Einstein's Mass--energy Equivalence]
\label{e=mc2}
The energy $E$ produced by a charge of mass $m$ is given by the mass-energy equivalence relation:
$$E=mc^2$$

where $c$ is the speed of light $= 299792458$ $ m s^{-1}$, $E$ is in Joules and $m$ is in kg.
\end{form}

The relation shows that mass is a form of energy. A small mass is equivalent to an enormous amount of energy.

\subsection{The Mass Defect}

When individual protons and neutrons come together to form the nucleus, there is a \textbf{decrease in mass}. i.e. The mass of the nucleus is smaller than the sum of masses of the individual nucleons. This decrease in mass is also known as the \textbf{mass defect}.

\begin{defi}[Mass Defect]
The mass defect of a nucleus is defined as the difference between the mass of the separated nucleons and the combined mass of the nucleus
\end{defi}

For example, to calculate the mass defect of a $^{40}_{20}Ca$ nucleus of mass $39.95159u$ : [\textit{mass of a proton $= 1.00728u$ ; mass of a neutron $= 1.00867 u$}]

\begin{center}
Mass of protons $= 20 \times 1.00728$\\
Mass of neutrons $= 20 \times 1.00867$\\
Total Mass $=40.3190u$\\
Mass Defect $d M = 40.3190u - 39.95159u = 0.36741u$
\end{center}
In general, to calculate the mass defect $d M$ for a \textbf{nucleus} that has \textbf{Z} protons and \textbf{N} neutrons,
\begin{form}[Calculating Mass Defect - Nucleus]
$d M = Zm_p + Nm_n -M_n$\\
Where:
$m_p$ = mass of a proton, 
$m_n$ = mass of a neutron, and 
$M_n$ = mass of the nucleus
\end{form}

As it is usually the masses of \textbf{neutral atoms} that are given, we can write the above equation as
\begin{form}[Calculating Mass Defect - Neutral Atoms]
$d M = Zm_p + Nm_n + Zm_e - M_n$\\
Where:
$m_e$ = mass of electron, and 
$M_n$ = mass of neutral atom
\end{form}

(Note: the second equation acutally gives the mass defect of the entire atom, and not only the nucleus.)

\subsection{The Binding Energy}
The mass of a nucleus is \textbf{always less} than the sum of the mass of its nucleons.
\begin{equation*}
\begin{split}
d M &= (Z m_p + n m_n) - m_{nucleus}\\
&= (Z m_p + N m_n) - (M_A - Z m_e)\\
&= Z M_H + N m_n - M_A\\
\end{split}
\end{equation*}
where Z is the proton number, N is the \textbf{neutron number}, i.e. the number  of neutrons in the nucleus (Numerically $A - Z$), $m_p$ the mass of proton, $m_n$ the mass of neutron, $m_e$ the mass of electron, $m_{nucleus}$ the actual mass of the nucleus, $M_n$ the mass of \textbf{hydrogen atom} and $M_A$ the atomic mass.
Previously, we learnt that nucleons inside the nucleus are held tightly together. Therefore, ENERGY is required to separate the nucleus into its constituent protons and neutrons.

\begin{defi}[Binding Energy]
The \textbf{binding energy of a nucleus} is the work which must be done on the nucleus to separate it completely into its constituent nucleons.
\end{defi}
Applying Einstein's mass-energy equivalence(Refer to \textit{Formula \ref{e=mc2}}), this shows that the binding energy of any nucleus is
\begin{form}[Calculating Binding Energy]
Binding Energy (B.E.) $ = ( d m) c^2$\\
Where $d m = $ mass defect and $c =$ the speed of light
\end{form}
Hence, the \textbf{binding energy of a nucleus} is the energy equivalent to the mass defect when nucleons bind together to form an atomic nucleus.

Combining with the fact that the mass of a nucleus is \textbf{\textit{always less}} than the sum of the masses of the nucleons, the binding energy of a nucleus is also the amount of energy released when nucleons band together to form an atomic nucleus.

At the atomic level,
\begin{enumerate}
\item the unit for mass is atomic mass unit(u), in place of kg.
\item the unit for energy is mega electron-volt, MeV, in place of J.
\end{enumerate}

\begin{center}
\begin{tabular}{ c c} \toprule
\multicolumn{2}{c}{SI units at subatomic levels}\\ \midrule \vspace{2pt}
Mass & $1u = 1.6605402 \times 10^{-27}kg$ \\
Energy & 1\textbf{ MeV} $= 1.60217733 \times 10^{-13}J$\\ \bottomrule
\end{tabular}
\end{center}

Thus,
\begin{equation*}
\begin{split}
1u &= 1.6605402 \times 10^{27} \times \frac{(2.99792458 \times 10^8)^2}{1.60217733 \times 10^{-23}}MeV\\
&= 931.494MeV
\end{split}
\end{equation*}
\paragraph{Calculation of Binding Energy}
\begin{tabbing}
Approach 1:\quad \= B.E. $ = (d m) c^2$\\
\>(find $d m$ in kg, use $m_p$, $m_n$ and $M_A$ expressed in \textbf{kg}, to obtain B.E. in J)\\
Approach 2: \> B.E. $= (Z m_p + Nm_n -M_A) \times 931.494 MeV$\\
\>(use $m_p$, $m_n$ and $M_A$ expressed in \textbf{u} to compute $d m$ to find B.E. in \textbf{MeV})
\end{tabbing}

\subsection{Binding Energy per Nucleon and Nuclear Stability}
The \textbf{binding energy(B.E.) per nucleon} of a nucleus is the \textbf{binding energy divided by the total number of nucleons}.
\begin{defi}[Binding energy per nucleon]
B.E. per nucleon $= \frac{B.E._{total}}{Nucleons}$
\end{defi}

This is also a measure of how stable the nucleus is -- the larger the binding energy per nucleon, the more stable.

The greater the binding energy per nucleon, the greater the work that must be done to remove the nucleon from the nucleus. Hence, a larger B.E. per nucleon indicates a stable nucleus.

The important features of the curve are:
\begin{itemize}
\item Except for the lighter nuclei, the \textbf{average binding energy per nucleon} is about 8 MeV.
\item The peak, i.e. the maximum B.E. per nucleon occurs at around mass number A = 50, and corresponds to the most stable nuclei. From the graph, it can be seen that the \textbf{iron nucleus} $^{56}_{26}Fe$ is located close to the peak with a B.E. per nucleon value of approximately 8.8MeV. It is one of the most stable nuclides that exist.
\item Nuclei with very low or very high mass numbers have lesser binding energy per nucleon and are less stable because the less B.E. per nucleon, the easier it is to separate the nucleus into its constituent nucleons.
\item Nuclei with low mass numbers may undergo \textbf{nuclear fusion}, where light nuclei are joined together under certain conditions so that the final product may have a greater binding energy per nucleon.
\item Nuclei with high mass numbers may undergo \textbf{nuclear fission} -- the nucleus may split into two daughter nuclei with the \textbf{release of neutrons}. The daughter nuclei will possess \textbf{a greater binding energy per nucleon.}
\end{itemize}

\begin{center}
\fbox{Energy released in reactions = $(m_{reactants} - m_{products})c^2 = (BE_{products} - BE_{reactants})$}
\end{center}
\subsection{Nuclear Reactions}
A nuclear reaction involves the rearrangement of nuclear constituents. \textbf{In all nuclear processes, the following quantities are conserved:}
\begin{itemize}
\item nucleon number
\item proton number(charge)
\item mass-energy
\item momentum
\end{itemize}

Nuclear reactions can be represented by an equation in which the total nucleon number and proton number balance on each side.

Example: \quad \quad $^1_0n + ^{14}_7N \rightarrow ^{14}_6 C +^1_1 p$

Nuclear reactions can also be represented by a symbolic expression of the form

Example: \quad \quad $^{14}_7 N (n,p) ^{14}_6 C$

Nuclear reactions can be classified into two types: \emph{spontaneous radioactive decay} and \emph{induced nuclear reactions}.

\subsection{Induced nuclear reactions}
Induced nuclear reactions occur when \emph{a nucleus changes as a result of being strucked by a particle}.

For example:
\begin{itemize}
\item Bombardment of nitrogen atoms N with $\alpha$--particles \emph{(Rutherford, 1919, refer to section \ref{rutherford})}
$$^{14}_7N + ^4_2He \rightarrow ^{17}_8 O + ^1_1H$$
\item Bombardment of beryllium atoms by $\alpha$--particles
$$_4^9Be + _2^4He \rightarrow _6^{12} C + _0^1 n$$
\end{itemize}

Note that in every reaction, the proton and nucleon number is conserved, i.e. equal on both sides of the reactions

\fbox{\parbox{\textwidth}{If the products have greater mass than the reactants(i.e. nucleus \& incident particle) before the reaction, than the incident particle must supply enough energy to make up for the increases in mass of the products to allow a reaction to take place}}

subsection{Nuclear fission}
Nuclear fission is the \textbf{disintegration of a heavy nucleus} into \textbf{two lighter nuclei of approximately equal masses}

Since the B.E. per nucleon of the daughters nuclei are higher than the parent's nucleus, energy is released.

For example:
\begin{itemize}
\item When Uranium--235 is bombarded by slow neutrons
$$_{92}^{235}U +_0^1 n \rightarrow _{56}^{142} Ba + _{36}^{91} Kr + 3(_0^1 n) + energy$$
\item The energy released by the fission of a single uranium--236 atom is about 200 MeV.
\item Most of this energy will appear in the form of the kinetic energies of the fission fragments and the neutrons, and in the form of the $\gamma$--radiation produced
\item Neutrons emitted would strike other Uranium--235 nuclei
\begin{itemize}
\item Collision of the neutrons produced in one fission reaction with other nuclei can give rise to a \emph{nuclear chain reaction}
\end{itemize}
\end{itemize}

\subsection{Nuclear Fusion}
\textbf{Nuclear fusion} is the \textbf{combining of two light nuclei to produce a heavier nucleus}

A large amount of energy is released during the process because the average binding energy per nucleon of the product has a greater binding energy per nucleon than the two light nuclei before fusion.

Fusion is a difficult process to achieve because of the strong electrical repulsion between the nuclei when they are close to each other. At extremely high temperatures (in excess of $10^8$ K) the nuclei have enough energy to overcome the repulsion.

An example is the fusion of two deuterium nuclei to produce helium--3:
$$_1^2H + _1^2 H \rightarrow _2^3He + _0^1 n$$

Reactions of this type(conversion of hydrogen to helium) are the source of the Sun's energy.

Energy released by the fusion of two nuclei is very much less than that which results from fission. (However, {fusion offers the possibility of energy from almost unlimited fuels, with the key advantage of non--radioactive waste.)

\subsection{Radioactivity}
\subsection{Nature of Radioactivity}
\begin{defi}[Radioactive decay]
Radioactive decay is the \textbf{spontaneous disintegration} of the nucleus of an atom which results in the \textbf{emission of particles} and/or \textbf{atom}
\end{defi}

\begin{itemize}
\item A radioactive nucleus consists of an unstable assembly of protons and neutrons which becomes more stable by emitting an $\alpha$, $\beta$ or $\gamma$ photon.
\item \textbf{Spontaneous Processes}
	\begin{itemize}
	\item Radioactive decay occurs \emph{spontaneously}. The process cannot be sped up or slowed down by physical means such as changes in pressure or temperature.
	\item The decay of a radioactive atom is \emph{not affected} by any chemical condition or the chemical compound it exists in and is independent of physical conditions such as temperature, pressure and most importantly the decay of other atoms.
	\end{itemize}
\item \textbf{Random Processes}
	\begin{itemize}
	\item Radiation is emitted at random. By random, we mean that it is impossible to predict which nucleus and when any particular nucleus will disintegrate.
	\end{itemize}
\end{itemize}

\subsection{Types of radiation}
\begin{tabular}{p{80pt} p{120pt} p{120pt} p{120pt} }\toprule
\hline
Property & $\alpha$--particles & $\beta$--particles & $\gamma$--particles \\ \midrule
Nature & Helium--4 nucleus & Electrons & Electromagnetic waves of short wavelength \\
Charge & +2e & -e & 0\\
Mass & $6.6464835 \times 10^{-27}$kg & $9.109387 \times 10^{-31}kg$ & 0 \\
Deflection by E and B fields & Deflected by strong fields & Deflected by weak fields & Undeflected\\
Energy & Constant for a given source & From zero up to a maximum, depending on source & Depends on frequency \\
Speed & around $10^7 ms^{-1}$ & around $10^8 ms^{-1}$ & $3.0 \times 10^8 ms^{-1}$\\
Range in Air & few cm & few metres & Follows the inverse square law\\
Ionising Power & Strong, producing $10^3$ to $10^4$ ions per mm of path & Less strong & Weak\\
Penetrating Power & $10^{-2}mm$ Aluminum & 5mm Aluminum & 100mm lead \\ \bottomrule
\end{tabular}

\begin{itemize}
\item Alpha Decay
\begin{itemize}
\item Alpha decays can be represented by the following equation, where \textbf{P} represents the parent nuclide and \textbf{D} the daughter nuclide)
$$_Z^A P \rightarrow _{Z-2}^{A-4} D + _2^4 He$$
\item Examples of alpha decays are:
$$_{92}^{238} U \rightarrow _{90}^{234} D + _2^4 He + \gamma$$
$$_{90}^{230} Th \rightarrow _{88}^{226} D + _2^4 He + \gamma$$
\end{itemize}
\item Beta Decay
\begin{itemize}
\item In beta decay the general equation is as follows:
$$_Z^A P \rightarrow _{Z+1}^A D + _{-1}^0 e $$
\item Examples of beta radiation are:
$$_{90}^{234} Th \rightarrow _{91}^{234} Pa + _{-1}^0 e + \gamma $$
$$_{91}^{234} Pa \rightarrow _{92}^{234} U + _{-1}^0 e + \gamma $$
\end{itemize}
\item Gamma Decay
\begin{itemize}
\item Gamma decay represents the \emph{\textbf{emission of energy}} from a nucleus which is returning to its ground state.
\item Excited nucleus $\rightarrow$ more stable nucleus +$\gamma$
\end{itemize}
\end{itemize}

\textbf{*Note: atomic and mass numbers are all conserved during all types of decay}

\subsection{Deflection of $\alpha$, $\beta$ and $\gamma$ in a strong magnetic field}

Above shows the paths of $\alpha$, $\beta$ and $\gamma$--particles/rays in a strong magnetic field.
\subsection{Activity, Half-life and Decay constant}
\begin{defi}[Radioactivity of a substance]
The activity of a radioactive substance is defined as the average number of atoms disintegrating per unit time
\end{defi}

\begin{itemize}
\item The \textbf{Activity A} or rate of decay of the parent nuclei is given by
\begin{form}[Radioactivity Equation 1]
$$A=-\frac{dN}{dt}$$
Where \textbf{N} is the number of nuclei, and \textbf{t} is the time. An activity of one decay per second is one Becquerel (1 Bq)
\end{form}
\item As radioactive decay is a random process, it follows the laws of statistics. The more radioactivity nuclei there are, the greater the activity
\item The activity \textbf{A} is directly proportional to the number of parent nuclei \textbf{N} at that instant
\begin{form}[Radioactivity Equation 2]
$$A \propto N$$
$$ A= -\frac{dN}{dt} = \lambda N$$
Where the constant of proportionality, $\lambda$, is called the decay constant, and it is the property of the particular radioactive nuclei.
\end{form}
\begin{defi}[Decay constant]
The decay constant of a nucleus is defined as its probability of decay per unit time\\[3pt]
Rationale:\qquad $\lambda = \frac{A}{N} = \frac{-\frac{dN}{dt}}{N} = \frac{-\frac{dN}{N}}{dt} = \frac{\text{ probability of decay}}{\text{time interval}}$
\end{defi}

\item Integrating, we get:
\begin{form}[Radioactivity Equation 3]
$$\ln N = \ln N_\circ = -\lambda t$$
Where $N_\circ$ is the initial number of radioactive nuclides and $N$ the number of nuclides remaining after a time t.
\end{form}

\item This gives:
$$N=N_\circ e^{-\lambda t}$$
\end{itemize}

Hence, it can be seen that the number of radioactive nuclides remaining after the time \textbf{t} decreases exponentially with time. This is the solution to the decay law.

\begin{itemize}
\item The above equation shows how the number of parent nuclei varies with time. It is an exponential decay.
\item The statistical uncertainty in a count of \textbf{N} is equal to $\sqrt{N}$. E.g. ($100 \pm 10$) counts in 1s gives uncertainty of 10\% while ($10000 \pm 100$) counts in 100s gives uncertainty of 1\%.
\item Therefore, $A=\lambda N = \lambda N_\circ e^{-\lambda t}=A_\circ e^{-\lambda t}$, where $A_\circ$ is the initial activity of the sample.
\end{itemize}

\subsection{Half-life}
\begin{defi}[Half-life]
Half-life is defined as the time taken for half the original number of radioactive nuclei to decay
\end{defi}

Substituting $N=\frac{1}{2}N_\circ$ and $t=t_{1/2}$ we get:
$$\frac{1}{2}N_\circ = N_\circ e^{-\lambda t_{1/2}}$$
$$2 = e^{\lambda t_{1/2}}$$
\begin{form}[Half-life formula]
$t_{1/2} = \frac{\ln 2}{\lambda} = \frac{0.693}{\lambda}$
\end{form}
