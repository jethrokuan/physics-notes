% !TEX root = ../Physics Notes.tex
\section{Current and Resistance}
\subsection{Electric Current $I$}
Whenever a charge is flowing, an electric current  said is to exist. The electric current, $I$, is defined as the \textbf{net amount of charge passing a point per unit time}. Hence, the \textit{magnitude of the current tells us the rate of the new flow of charge}.

The current is the rate at which charge flows through this surface. If $d Q$ is the amount of charge that passes through the surface in a time interval $d t$, the \textbf{average electric current $I_{ave}$} is equal to the charge that passes through A per unit time.

\begin{defi}[Average Current]
$$I = \frac{d Q}{d t}$$
Where $Q$ is charge and $t$ is time.
\end{defi}

It is sometimes, even very often, useful to describe current in terms of current density. Current density is the amount of "current" per unit area. This quantity is rather important in the derivations for the magnetic field chapter. We denote current density by $J$.

\begin{defi}[Current Density]
$$J = \frac{I}{A}$$
Where $I$ is current, and $A$ is area.
\end{defi}