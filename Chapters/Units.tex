\section{Units and Physical Quantities}
Physics is a fundamental science. The study of physics is both challenging and interesting, though occasionally frustrating.

Physics allows you to understand physical phenomena around you, such as mirages etc.

You must first understand that physics is not all about theory. In fact, it is experimental science, and physicists determine formulas based on many assumptions, mostly reasonable, to estimate and predict phenomena.

A theory is first derived, and experimentally shown to hold true. If any evidence shows that this theory does not hold, it is modified, to hold true again. This process repeats until the theory is solid enough to hold for almost all phenomena.

As such, we see that a theory has to be falsifiable. Otherwise this is not science.

\subsection{Solving Problems}
In their study of Physics, many people complain that they understand the concepts, but are not able to solve problems, or investigate the causes of phenomena. The key lies in \textit{applying} these concepts.

I shall adopt the use of the acronym and method of problem solving from Young and Freedman, which is \textbf{I SEE}. Identify, Set up, Execute, Evaluate.

\subsection{Identify}
Firstly, we must identify the assumptions we have to make in order to solve the problem. For example, if we were to analyse the motion of a ball, we probably assume that air drag is negligible, or constant against speed for a more accurate result. If we were to consider air resistance, then the equations would become really messy.

A key part of physics is simplification. Physicists believe in simplicity. And it is based on this belief that they have been driven to work on finding a theory of everything, and they have come close, proposing a superstring theory(though still with many loopholes, for example having 11 dimensions instead of 4)

We must first identify that the problem deals with the motion of a particle, and thus classify this as a kinematics problem. We will make the assumption that the ball is a point mass to simplify our question.