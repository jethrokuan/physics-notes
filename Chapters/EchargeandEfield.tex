% !TEX root = ../Physics Notes.tex

\section{Electric Charge and Electric Field Calculations}

In this chapter, we proceed with the study of charges, and the fields they produce. Fields might not appear all new to you. This is because we have also dealt with the gravitation field theory earlier.

\subsection{Electric Charge}
So what exactly is an electric charge? How do we know that it exists in the first place?

Some simple experiments demonstrate the existence of electric forces. For example, after rubbing a balloon on your hair on a dry day, we will find that the balloon would attract pieces of paper. The balloon is said to have been charged, and this electric force is actually the electric force of attraction.

When such materials exhibit this behaviour, they are said to be \emph{electrified} or to have become \textbf{electrically charged}. One can easily electrify himself by rubbing his/her shoes against a cloth rug.

Using this series of experiments, \textbf{Benjamin Franklin} found that there were two kinds of electric charges, which were given the names \textbf{positive} and \textbf{negative}. Electrons have been identified to possess \textbf{negative} charge, while protons possess \textbf{positive} charge. \textbf{Charges of the same sign repel one another, while charge sof the opposite sign attract one another}.

One observation of much importance is the electrical charge is conserved in an isolated system. That is, when one object is rubbed against another, charge is not created in the process. Instead \textbf{charge is transferred} from one object to another, and result in both objects having an imbalance of charge.

Later on, Robert Millikan discovered that electric charge was a quantised. electric charge, $q$, was always in integer multiples of the fundamental amount of charge $e$. Therefore, we can write $q = \pm Ne$ where $N$ is some integer.

\subsection{Induction}
It is important and convenient that we are able to classify materials according to their ability of electrons to move around the material. This is the classification of conductors, insulators, and the "in-betweens".

\begin{center}
\fbox {
    \parbox{0.9\linewidth}{
     Electrical conductors are materials in which some of the electrons are free electrons that are not bound to atoms and can move relatively freely through the material; electrical insulators are materials in which all electrons are bound to atoms and cannot move freely through the material.
    }
}
\end{center}

