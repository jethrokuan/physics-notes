\documentclass[11pt]{article} 
\usepackage{amsmath} % AMS Math Package
\usepackage{amsthm} % Theorem thematting
\usepackage{amssymb}	% Math symbols such as \mathbb
\usepackage{graphicx} % Allows for eps images
\usepackage{hyperref} % link contents page
\usepackage{multicol} % Allows for multiple columns
\usepackage{booktabs}

\usepackage[margin=0.75in]{geometry}

 % Sets margins and page size
\renewcommand{\labelenumi}{(\alph{enumi})} % Use letters for enumerate
% \DeclareMathOperator{\Sample}{Sample}
\let\vaccent=\v % rename builtin command \v{} to \vaccent{}

\renewcommand{\v}[1]{\ensuremath{\mathbf{#1}}} % for vectors
\newcommand{\gv}[1]{\ensuremath{\mbox{\boldmath$ #1 $}}} 
% for vectors of Greek letters
\newcommand{\uv}[1]{\ensuremath{\mathbf{\hat{#1}}}} % for unit vector
\newcommand{\abs}[1]{\left| #1 \right|} % for absolute value
\newcommand{\avg}[1]{\left< #1 \right>} % for average
\let\underdot=\d % rename builtin command \d{} to \underdot{}
\renewcommand{\d}[2]{\frac{d #1}{d #2}} % for derivatives
\newcommand{\dd}[2]{\frac{d^2 #1}{d #2^2}} % for double derivatives
\newcommand{\pd}[2]{\frac{\partial #1}{\partial #2}} 
% for partial derivatives
\newcommand{\pdd}[2]{\frac{\partial^2 #1}{\partial #2^2}} 
% for double partial derivatives
\newcommand{\pdc}[3]{\left( \frac{\partial #1}{\partial #2}
 \right)_{#3}} % for thermodynamic partial derivatives

\newcommand{\ket}[1]{\left| #1 \right>} % for Dirac bras
\newcommand{\bra}[1]{\left< #1 \right|} % for Dirac kets
\newcommand{\braket}[2]{\left< #1 \vphantom{#2} \right|
 \left. #2 \vphantom{#1} \right>} % for Dirac brackets

\newcommand{\matrixel}[3]{\left< #1 \vphantom{#2#3} \right|
 #2 \left| #3 \vphantom{#1#2} \right>} % for Dirac matrix elements
\newcommand{\grad}[1]{\gv{\nabla} #1} % for gradient
\let\divsymb=\div % rename builtin command \div to \divsymb
\renewcommand{\div}[1]{\gv{\nabla} \cdot #1} % for divergence
\newcommand{\curl}[1]{\gv{\nabla} \times #1} % for curl
\let\baraccent=\= % rename builtin command \= to \baraccent

\newcommand{\cen}[1]{\begin{center}{#1}\end{center}}

\newcommand{\eqn}[1]{\begin{equation}{#1}\end{equation}}
\newcommand{\noeqn}[1]{\begin{equation*}{#1}\end{equation*}}
\newcommand{\spliteqn}[1]{\begin{equation}\begin{split}{#1}\end{split}\end{equation}}
\newcommand{\nospliteqn}[1]{\begin{equation*}\begin{split}{#1}\end{split}\end{equation*}}
\newcommand{\C}[2] {\left(\begin{array}{c} #1 \\ #2 \end{array}\right)}
\newcommand{\brac}[1] {\left(#1 \right)}

\newtheorem{form}{Theorem}[section]
\newtheorem{lem}[form]{Lemma}
\theoremstyle{definition}
\newtheorem{defi}{Definition}
\theoremstyle{plain}% default

\setlength{\parskip}{0.1in}
\setlength{\unitlength}{1mm}